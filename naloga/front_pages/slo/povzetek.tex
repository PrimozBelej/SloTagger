%---------------------------------------------------------------
% SLO: slovenski povzetek
% ENG: slovenian abstract
%---------------------------------------------------------------
\selectlanguage{slovene} % Preklopi na slovenski jezik
\addcontentsline{toc}{chapter}{Povzetek}
\chapter*{Povzetek}

\noindent\textbf{Naslov:} \ttitle
\bigskip

V magistrskem delu se ukvarjamo z oblikoskladenjskim označevanjem slovenskega jezika. Pri tej nalogi s področja obdelave naravnega jezika povedim priredimo ustrezno zaporedje oznak, ki opisujejo oblikoskladenjske lastnosti besed. Za razliko od tipičnih pristopov, ki vhodne povedi obravnavajo na nivoju besed, naša rešitev obravnava vhodne povedi kot zaporedja znakov. Nalogo označevanja rešujemo s kombinacijo konvolucijskih in rekurentnih nevronskih mrež. Posebnost našega pristopa je tudi v sami naravi označevanja, saj ga ne obravnavamo kot problem večrazredne klasifikacije, temveč kot večznačno klasifikacijo, kjer primerom dodeljujemo oznake. Z namenom izboljšave rezultatov našo rešitev združimo v ansambel treh označevalnikov, skupaj z dvema obstoječima označevalnikoma za slovenski jezik. Ob primerjavi naše rešitve z obstoječimi ugotovimo, da predlagana rešitev dosega najboljše rezultate pri reševanju zadanega problema.

\subsection*{Ključne besede}
\textit{\tkeywords}
\clearemptydoublepage
